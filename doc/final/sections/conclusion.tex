\section{Conclusion}
\label{sec:conclusion}

We investigated MAPPO for power grid frequency control with 10 heterogeneous agents, revealing both potential and fundamental limitations of decentralized MARL.

\textbf{What worked:} Critic loss converged ($10^{13} \rightarrow 22$), agents learned smooth control (wear costs halved), episodes reached 350--400 steps, and peak performance achieved reward $\sim$165 around episode 1000.

\textbf{What didn't:} Policy performance degraded from $\sim$165 to $\sim$120 in later training, exhibiting the ``forgetting'' phenomenon despite continued critic convergence.

\subsection{The Fundamental Challenge}

Coordinating 10 independent agents under partial observability faces inherent obstacles: non-stationarity (other agents change the environment), credit assignment (shared rewards obscure individual contribution), and theoretical hardness (Dec-POMDP is NEXP-complete~\cite{bernstein2002}). The decreasing critic loss alongside degrading policy is key: value learning succeeded, but coordinating policy updates across 10 agents failed.

\subsection{Lessons Learned}

\begin{enumerate}[nosep]
    \item \textbf{Physical feasibility first}---verify optimal policy can succeed before training
    \item \textbf{Value learning $\neq$ policy learning}---critic convergence doesn't guarantee actor convergence in MARL
    \item \textbf{Coordination is fragile}---individual policy updates can break discovered coordination
    \item \textbf{MAPPO has limits}---tight multi-agent coordination may require value decomposition, communication, or hierarchy
\end{enumerate}

\subsection{Future Work}

Extensions to address coordination: QMIX/VDN for credit assignment~\cite{rashid2020}, attention mechanisms, explicit communication channels, hierarchical control, and comparison with PI-AGC baselines.

\textbf{Broader implication:} Multi-agent coordination is fundamentally hard. The independence enabling decentralized execution also makes coordinated learning unstable. Reliable MARL for safety-critical infrastructure likely requires structured approaches rather than purely independent agents.
