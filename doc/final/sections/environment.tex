\section{Environment Design}
\label{sec:environment}

\subsection{Grid Topology and Agents}

We implement a 20-bus power network with 10 heterogeneous agents: 2 batteries (50 MW/min, [0,100] MW), 5 gas plants (10 MW/min, [50,500] MW), and 3 demand response units (5 MW/min, [-200,0] MW). Seven renewable sources provide 50--300 MW each with stochastic variation.

\textbf{Capacity matching} is critical: total controllable generation ($\sim$3300 MW) must exceed load range. We set loads to [1500, 3000] MW to ensure agents can always balance the grid---without this, learning is impossible.

\subsection{Physics and Dynamics}

Frequency dynamics follow the \textbf{swing equation}:
\begin{equation}
\frac{df_k}{dt} = \frac{P_{\text{mech},k} - P_{\text{elec},k}}{2H_k \cdot S_{\text{base}}}
\end{equation}
with inertia $H_k \in [2, 7]$s, base power $S_{\text{base}} = 10,000$ MVA, time step $\Delta t = 2$s (SCADA delay), and N-1 contingencies ($p=0.001$/step).

\subsection{State and Action Spaces}

\textbf{Local observation (15-dim):} frequency deviation, load, own output, system frequency deviation, 5 neighbor frequencies, 3-step renewable forecast, time features, capacity utilization.

\textbf{Global state (55-dim):} All bus frequencies, generator outputs, renewables, loads, time features.

\textbf{Actions:} Continuous $[-1, 1]$ scaled by agent-specific ramp rates.

\subsection{Curriculum Learning}

We progressively tighten frequency bounds: Stage 1 (ep 0--1500): $\pm 2.5$ Hz; Stage 2: $\pm 2.2$ Hz; Stage 3: $\pm 2.0$ Hz; Stage 4 (ep 3500+): $\pm 1.8$ Hz. This allows agents to learn basic control before facing strict constraints.
