\section{Environment Design}
\label{sec:environment}

\subsection{Grid Topology}

We implement a simplified power grid environment with the following components:
\begin{itemize}[nosep]
    \item \textbf{20-bus power network} with realistic topology constraints and admittance matrix
    \item \textbf{10 heterogeneous controllable agents:}
    \begin{itemize}[nosep]
        \item 2 Battery storage units: Fast response (50 MW/min), capacity [0, 100] MW each
        \item 5 Gas plants: Slower ramp rates (10 MW/min), capacity [50, 500] MW each
        \item 3 Demand response units: Load shedding capability (5 MW/min), range [-200, 0] MW each
    \end{itemize}
    \item \textbf{7 renewable generation sources} with stochastic output (50--300 MW each)
    \item \textbf{Distributed loads} with total system load in [1500, 3000] MW range
\end{itemize}

\textbf{Capacity Analysis.} A critical design consideration is ensuring agents have sufficient capacity to balance the grid:
\begin{center}
\begin{tabular}{lcc}
\toprule
\textbf{Component} & \textbf{Min (MW)} & \textbf{Max (MW)} \\
\midrule
Batteries (2$\times$) & 0 & 200 \\
Gas Plants (5$\times$) & 250 & 2500 \\
Demand Response (3$\times$) & -600 & 0 \\
Renewables (7$\times$) & 140 & 2100 \\
\midrule
Total Controllable & -- & $\sim$3300 \\
\bottomrule
\end{tabular}
\end{center}

The load range [1500, 3000] MW ensures agents can always balance the grid, avoiding scenarios where control is physically impossible.

\subsection{Physics Model}

The environment uses the \textbf{swing equation} to model frequency dynamics at each bus $k$:
\begin{equation}
\frac{df_k}{dt} = \frac{P_{\text{mech},k} - P_{\text{elec},k}}{2H_k \cdot S_{\text{base}}}
\label{eq:swing}
\end{equation}
where $H_k \in [2, 7]$ seconds is the inertia constant, $S_{\text{base}} = 10,000$ MVA is the base power, and the power imbalance determines frequency deviation from 60 Hz.

Key dynamics features:
\begin{itemize}[nosep]
    \item \textbf{Time step:} $\Delta t = 2$ seconds (SCADA delay)
    \item \textbf{Communication delay:} 1 time step observation buffer
    \item \textbf{N-1 contingencies:} Random bus disconnection with probability 0.001/step
    \item \textbf{No frequency clamping:} Physics runs naturally for realistic dynamics
\end{itemize}

\subsection{Observation and Action Spaces}

\textbf{Local Observation per Agent (15 dimensions):}
\begin{enumerate}[nosep]
    \item Local bus frequency deviation (scaled by curriculum bound)
    \item Local bus load (normalized)
    \item Own generator output (normalized by capacity)
    \item System-wide frequency deviation (coordination signal)
    \item Nearby bus frequency deviations (5 neighbors)
    \item Renewable generation forecasts (3 time steps ahead)
    \item Time features (hour, day of week)
    \item Own capacity utilization
\end{enumerate}

\textbf{Global State for Critic (55 dimensions):}
All bus frequencies, generator outputs, renewable generation, loads, and time features.

\textbf{Actions:} Continuous power adjustments in $[-1, 1]$, scaled by agent-specific ramp rate limits.

\subsection{Curriculum Learning}

To facilitate learning, we implement a 4-stage curriculum that gradually tightens frequency tolerances:

\begin{center}
\begin{tabular}{ccccc}
\toprule
\textbf{Stage} & \textbf{Episodes} & \textbf{Critical (Hz)} & \textbf{Catastrophic (Hz)} & \textbf{Threshold} \\
\midrule
1 & 0--1500 & $\pm 2.5$ & $\pm 3.5$ & 30\% \\
2 & 1500--2500 & $\pm 2.2$ & $\pm 3.2$ & 28\% \\
3 & 2500--3500 & $\pm 2.0$ & $\pm 3.0$ & 25\% \\
4 & 3500+ & $\pm 1.8$ & $\pm 2.5$ & 20\% \\
\bottomrule
\end{tabular}
\end{center}

This allows agents to first learn basic survival and power balancing before facing stricter operational constraints.

