\section{Introduction}
\label{sec:introduction}

\subsection{Motivation}

Modern power grids face increasing complexity due to renewable energy integration, which introduces stochastic generation patterns. With over 30\% renewable penetration in many regions, grids experience reduced inertia and doubled rate-of-change-of-frequency events, leading to \$10B+ annual regulation costs~\cite{nerc2023}. Traditional centralized control approaches---including PI controllers and model predictive control---struggle to scale with grid complexity and cannot effectively optimize multi-step costs under stochastic disturbances~\cite{kundur1994,venkat2008}.

Multi-agent reinforcement learning (MARL) offers a promising alternative for coordinated, adaptive control with potential 20--40\% cost reduction~\cite{venkat2022}. This project explores \textbf{Centralized Training, Decentralized Execution (CTDE)} using Multi-Agent Proximal Policy Optimization (MAPPO), where:
\begin{itemize}[nosep]
    \item \textbf{Training:} A centralized critic observes global state to estimate value functions
    \item \textbf{Execution:} Each agent acts independently using only local observations
\end{itemize}

This paradigm enables scalable, real-time decision-making while leveraging global information during learning.

\subsection{Problem Formulation}

We formulate frequency regulation as a cooperative Multi-Agent Markov Decision Process (MA-MDP) with $N=10$ heterogeneous agents across a 20-bus network:

\begin{enumerate}[nosep]
    \item \textbf{Decision-makers:} 2 batteries (50 MW/min ramp rate), 5 gas plants (10 MW/min), 3 demand response units (5 MW/min)
    \item \textbf{Dynamics:} Swing equation governing frequency evolution (Section~\ref{sec:environment})
    \item \textbf{Sequential nature:} Multi-step lookahead required due to renewable forecasts, load fluctuations, communication delays (2s), and safety constraints ($\pm 0.5$ Hz operational bounds)
\end{enumerate}

\subsection{Challenges}

This problem presents several fundamental challenges:
\begin{itemize}[nosep]
    \item \textbf{Continuous spaces:} State $\mathcal{S} \subseteq \mathbb{R}^{55}$, local observations $\mathcal{O}^i \in \mathbb{R}^{15}$, actions $\mathcal{A}^i \in \mathbb{R}$
    \item \textbf{Partial observability:} Agents observe only local bus frequencies and limited neighbor information
    \item \textbf{Stochastic disturbances:} Renewable generation and load variations
    \item \textbf{Hard safety constraints:} Frequency must remain within $\pm 1.5$ Hz to avoid cascading failures
    \item \textbf{Multi-agent coordination:} Non-stationarity, credit assignment, and scalability
    \item \textbf{Capacity constraints:} Physical limits on generation and ramp rates
\end{itemize}

\subsection{Contributions}

This project makes the following contributions:
\begin{enumerate}[nosep]
    \item A complete 20-bus power grid simulator with realistic swing equation dynamics
    \item MAPPO implementation with CTDE for heterogeneous agent coordination
    \item Systematic debugging methodology for reward scaling, capacity matching, and curriculum learning
    \item Comprehensive failure mode analysis and training diagnostics
    \item Open-source codebase with TensorBoard integration for reproducibility
\end{enumerate}

% TODO: Add final performance summary once training completes

